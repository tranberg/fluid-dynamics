\documentclass[a4paper, 11pt]{article}

\usepackage[T1]{fontenc}
\usepackage[utf8]{inputenc}
\usepackage[english]{babel}
\usepackage{microtype}
\usepackage[margin=3cm]{geometry}
\usepackage{lipsum}
\usepackage{url}
\usepackage{graphicx}
\usepackage{cite}
\usepackage[hidelinks]{hyperref}
\usepackage{amsmath}
\usepackage[draft]{fixme}
\usepackage{framed}
%\definecolor{shadecolor}{rgb}{1,0.8,0.3}

\usepackage[sc]{mathpazo}
\linespread{1.05}
\parindent 0pt
\parskip 4pt

\usepackage{booktabs}
\setlength{\heavyrulewidth}{0.15em}
\setlength{\lightrulewidth}{0.08em}

% Where to look for figures
\graphicspath{{./figures/}}

% Macros
\newcommand{\sref}[1]{Section~\ref{#1}}
\newcommand{\fref}[1]{Figure~\ref{#1}}
\newcommand{\tref}[1]{Table~\ref{#1}}
\renewcommand{\eqref}[1]{(\ref{#1})}


% Title stuff
\title{Note for Fluid Dynamics Midterm Exam Project I}
\author{Bo Tranberg\ \ \ Kun Zhu \\\href{mailto:bo@eng.au.dk}{bo@eng.au.dk}\href{mailto:kunzhu@eng.au.dk}\ \ \ {kunzhu@eng.au.dk}}
\date{\today}

%----------------------------------------------------------------------------------------

\begin{document}
\maketitle

\large{
	\textbf{Remark}: This is an auxiliary note which you are free to choose use or not for the midterm exam project I. 
	}

\section{Coordinate rotation}
\begin{framed}
	Explain a bit why we need to rotate the coordinate.
\end{framed}
\begin{equation}
	\begin{pmatrix}
	x_r \\
	y_r
	\end{pmatrix}
	=
	\begin{pmatrix}
	\cos \theta \ \ \  \sin \theta\\
	-\sin \theta \ \ \cos \theta
	\end{pmatrix}
	\begin{pmatrix}
	x \\
	y
	\end{pmatrix}
\end{equation}
where $(x,y)$ are the original coordinates of the turbines, and $(x_r,y_r)$ are the corresponding rotated coordinates.
\section{Overlapping area}
We denote the radius of rotor as $r_2$, and the radius of wake at turbine we are looking at as $r_1$. If the distance $d$ between the center of turbine and wake is larger than $(r_1+r_2)$, i.e.,
\begin{equation}
d \geq r_1+r_2
\end{equation}
then we end up no overlap between the turbine and wake. When $d < r_1+r_2$ there begins overlapping, but we need to distinguish between two different cases. If $d$ satisfies the following equation, 
\begin{equation}
r_1-r_2 < d < r_1+r_2
\end{equation}
we have the partially overlap case, where is overlapping area can be found as
\begin{equation}
area = r^2_1 \cos^{-1}\bigg(\frac{r^2_1-r^2_2+d^2}{2dr_1}\bigg)
      +r^2_2 \cos^{-1}\bigg(\frac{r^2_2-r^2_1+d^2}{2dr_2}\bigg)
      -\frac{1}{2}\sqrt{T}
\end{equation}
where $T$ can be calculated as 
\begin{equation}
T = \Big((r_1+r_2)^2-d^2\Big)\Big(d^2-(r_1-r_2)^2 \Big)
\end{equation}
The last case is fairly straightforward, where the turbine is fully covered by the wake, i.e.,
\begin{equation}
	\begin{aligned}
	d \leq r_1-r_2 \\
	area = \pi r^2_2
	\end{aligned}
\end{equation}

\section{Wake deflection due to yaw}
\begin{framed}
	What do we need to add here?
\end{framed}
\end{document}
 